%----------------------------------------------------------------------------------------
%	PACKAGES AND THEMES
%----------------------------------------------------------------------------------------

\documentclass[table]{beamer}

\mode<presentation> {

% The Beamer class comes with a number of default slide themes
% which change the colors and layouts of slides. Below this is a list
% of all the themes, uncomment each in turn to see what they look like.

%\usetheme{default}
%\usetheme{AnnArbor}
%\usetheme{Antibes}
%\usetheme{Bergen}
%\usetheme{Berkeley}
%\usetheme{Berlin}
%\usetheme{Boadilla}
%\usetheme{CambridgeUS}
\usetheme{metropolis}
%\usetheme{Copenhagen}
%\usetheme{Darmstadt}
%\usetheme{Dresden}
%\usetheme{Frankfurt}
%\usetheme{Goettingen}
%\usetheme{Hannover}
%\usetheme{Ilmenau}
%\usetheme{JuanLesPins}
%\usetheme{Luebeck}
%\usetheme{Madrid}
%\usetheme{Malmoe}
%\usetheme{Marburg}
%\usetheme{Montpellier}
%\usetheme{PaloAlto}
%\usetheme{Pittsburgh}
%\usetheme{Rochester}
%\usetheme{Singapore}
%\usetheme{Szeged}
%\usetheme{Warsaw}

% As well as themes, the Beamer class has a number of color themes
% for any slide theme. Uncomment each of these in turn to see how it
% changes the colors of your current slide theme.

%\usecolortheme{albatross}
%\usecolortheme{beaver}
%\usecolortheme{beetle}
%\usecolortheme{crane}
%\usecolortheme{dolphin}
%\usecolortheme{dove}
%\usecolortheme{fly}
%\usecolortheme{lily}
%\usecolortheme{orchid}
%\usecolortheme{rose}
%\usecolortheme{seagull}
%\usecolortheme{seahorse}
%\usecolortheme{whale}
%\usecolortheme{wolverine}

%\setbeamertemplate{footline} % To remove the footer line in all slides uncomment this line
%\setbeamertemplate{footline}[page number] % To replace the footer line in all slides with a simple slide count uncomment this line

%\setbeamertemplate{navigation symbols}{} % To remove the navigation symbols from the bottom of all slides uncomment this line
}

\usepackage{graphicx} % Allows including images
\usepackage{epstopdf}
\usepackage{booktabs} % Allows the use of \toprule, \midrule and \bottomrule in tables
\usepackage{multirow}
\usepackage[export]{adjustbox}
\usepackage{gensymb}
\renewcommand\footnoterule{{\color{black}\hrule height 0pt}}
\usepackage{appendixnumberbeamer} 
\usepackage{textgreek}
\usepackage{xcolor}
\usepackage{colortbl}
\usepackage{gensymb}
\graphicspath{{pics}}
\usepackage[normalem]{ulem}
\usepackage{mathtools}
\usepackage{amsmath}
\usepackage{amsfonts}

\newcommand{\overbar}[1]{\mkern 1.5mu\overline{\mkern-1.5mu#1\mkern-1.5mu}\mkern 1.5mu}

\setbeamertemplate{caption}{\raggedright\insertcaption\par}
\setbeamertemplate{navigation symbols}{}
\setbeamertemplate{itemize items}[circle]

\usepackage[absolute,overlay]{textpos}

%\AtBeginSection{\frame{\sectionpage}}

%\definecolor{grayprimary}{rgb}{0.77,0.84,0.94}
%\definecolor{lbnldarkblue}{RGB}{0,49,60}
%\definecolor{lbnlteal}{RGB}{0,118,129}
%\definecolor{lbnldarkgray}{RGB}{99,102,106}
%
%
%\setbeamercolor*{palette primary}{use=structure,fg=white, bg=lbnldarkgray}
%\setbeamercolor*{palette secondary}{use=structure,fg=white,bg=lbnlteal}
%\setbeamercolor*{palette tertiary}{use=structure,fg=white,bg=lbnldarkblue}
%\setbeamercolor{title}{fg=lbnldarkblue}
%%\setbeamercolor*{palette tertiary}{use=structure,fg=white,bg=green}
%\setbeamercolor{frametitle}{fg=lbnldarkblue}
%\setbeamercolor{author}{fg=lbnldarkblue}
%\setbeamercolor{date}{fg=lbnldarkblue}
%
%\setbeamertemplate{page number in head/foot}[totalframenumber]

%----------------------------------------------------------------------------------------
%	TITLE PAGE
%----------------------------------------------------------------------------------------

\title[Group Meeting]{Isotope effects - towards isotope chemistry} % The short title appears at the bottom of every slide, the full title is only on the title page

\author{Fatima H. Garcia}
\institute[LBNL] % Your institution as it will appear on the bottom of every slide, may be shorthand to save space
{} % Your email address  % Date, can be changed to a custom date
\date{April 21, 2022}



\begin{document}
\begin{frame}
\titlepage % Print the title page as the first slide
%\begin{figure}
%\includegraphics[width=0.5\textwidth]{logos.png}
%\end{figure}
\end{frame}
 
 

%----------------------------------------------------------------------------------------
%	PRESENTATION SLIDES
%----------------------------------------------------------------------------------------

%------------------------------------------------

\section{Definitions}
\begin{frame}
\frametitle{Isotope effects}
\textbf{Kinetic isotope effect}:\\
Change in chemical reaction rates when the atoms is replaced with  different isotope
\begin{itemize}
\item Can be used to determine preferable reaction pathways for optimization
\item Change is most pronounced with large differences - effect is related to vibration of the bonds
\end{itemize}
\end{frame}

\begin{frame}
\frametitle{Isotope effects}
\textbf{Isotope fractionation}:\\
Relative partitioning or distribution of isotopes in a natural system based on their mass
\begin{itemize}
\item \textbf{Mass/Kinetic}: separation is due to mass differences; in biology, organisms prefer lighter isotopic species
\item \textbf{Equilibrium/Thermodynamic}: species in chemical eq.; due to reduction in vibrational energy in the substitution of a heavier isotope with a lighter one
\item \textbf{Transient kinetic}: reactions do not follow first-order reaction rates
\item \textbf{Mass-independent}: differences are not correlated to mass differences; non-equilibrium processes
\end{itemize}
\end{frame}


\section{Literature cases}

\begin{frame}
\frametitle{Literature division}
\begin{columns}[c]
\column{0.5\textwidth}
\textbf{Biology}
\begin{itemize}
\item Focus is on Ca, H/D and O
\item Mostly plants, though some bacteria have been studied
\item Fractionation is attributed to transfer processes
\end{itemize}
\column{0.5\textwidth}
\textbf{Geology/geochemistry}
\begin{itemize}
\item Use of 'standard' samples
\item Use of 'surrogates'
\item Different metals are studied in the context of mineral formation
\end{itemize}
\end{columns}
\end{frame}

\begin{frame}
\frametitle{Common themes in the literature}
\begin{itemize}
\item Lighter isotopes make weaker bonds within molecules
\item 'Standards' that are used are very much not standard - its all comparative
\item Too many factors that would need controls for characterization
\item Processes for fractionation are not well understood in geology/geochemistry
\item Use of fractionation to infer environmental conditions such as [O$_{2}$]
\end{itemize}
\end{frame}


\begin{frame}
\frametitle{Fractionation and Wine}
A study was done to use NMR techniques to perform analysis on wine. \\
They note some interesting things:
\begin{itemize}
\item Enrichment techniques had previously been used for studies of photosynthesis
\item Info about the redistribution of H isotopes in the process of turning glucose and must into ethanol and water
\item Constant distribution of isotope 'redistribution' matrix allowed for 'fingerprinting' for a specific region
\begin{itemize}
\item Isotope ratios are of climatic significance
\end{itemize}
\end{itemize} 
Word of the day: \\
\textbf{Isotopemer}: \textit{noun} isotopic isomer; same number of each isotope with the molecule but at different positions in the structure
\vspace{10pt}
\let\thefootnote\relax\footnote{\tiny{Martin, G. J. \textit{et al.}. J. Agric. Food Chem. 36, 2 (1988) 316-322}}
\end{frame}


\begin{frame}
\frametitle{Calcium}
\small{
Hart and Zindler:
\vspace{-4pt}
\begin{itemize}
\item Proposed that a 'mixing' process during fractionation is the cause of deviations between experiment and theory
\end{itemize}
\vspace{-4pt}
Hindshaw \textit{et al.}:
\vspace{-4pt}
\begin{itemize}
\item Significant fractionation is observed in soil and root samples 
\item Age plays a factor, since nutrient uptake changes with time
\item More fractionation in the roots than the stems
\item Incomplete kinetic reactions favour lighter Ca isotopes
\end{itemize}
\vspace{-4pt}
Wang \textit{et al.}:
\vspace{-4pt}
\begin{itemize}
\item  The [M] has an effect on O bond lengths and may be tied to fractionation
\end{itemize}}
\vspace{-10pt}
\let\thefootnote\relax\footnote{\tiny{Hart, S. R. and Zindler, A. Int. J. Mass Spectrom. Ion Pro. 89 (1989) 287-301}}
\let\thefootnote\relax\footnote{\tiny{Hindshaw, R. S. \textit{et al.} Biogeochemistry 112 (2013) 373-388}}
\let\thefootnote\relax\footnote{\tiny{Wang, W. \textit{et al.} Geochim. Cosmochim. Acta 208 (2017) 185-197}}
\end{frame}

\begin{frame}
\frametitle{Uranium}
Andersen \textit{et al.}:
\begin{itemize}
\item Difference in solubility of the charge states will impact fractionation
\item Heavier U isotopes favour compounds with lower oxidation states
\begin{itemize}
\item Opposite to mass dependent effects, which are due to vibrations
\end{itemize}
\item Studies with no U redox show limited fractionation
\end{itemize}
Yang \& Liu:
\begin{itemize}
\item Nuclear field shift effect may decrease with temperature - also called the nuclear volume effect (?)
\item Effect scales with the difference of mean-square charge radius
\end{itemize}
\vspace{-5pt}
\let\thefootnote\relax\footnote{\tiny{Andersen, M. B, Stirling, C. H. and Weyer, S. Rev. Mineral Geochem. 82 (2017) 799-850}}
\let\thefootnote\relax\footnote{\tiny{Yang, S. and Liu, Y. Acta Geochim. 35(3) (2016) 227-239}}
\end{frame}

%
%%------------------------------------------------
%

%
%\appendix
%
%\begin{frame}
%\frametitle{Beginning of backup slides}
%\end{frame}


%-----------------------------------------------

\end{document} 
