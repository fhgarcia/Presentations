%----------------------------------------------------------------------------------------
%	PACKAGES AND THEMES
%----------------------------------------------------------------------------------------

\documentclass[table]{beamer}

\mode<presentation> {

% The Beamer class comes with a number of default slide themes
% which change the colors and layouts of slides. Below this is a list
% of all the themes, uncomment each in turn to see what they look like.

%\usetheme{default}
%\usetheme{AnnArbor}
%\usetheme{Antibes}
%\usetheme{Bergen}
%\usetheme{Berkeley}
%\usetheme{Berlin}
%\usetheme{Boadilla}
\usetheme{CambridgeUS}
%\usetheme{Copenhagen}
%\usetheme{Darmstadt}
%\usetheme{Dresden}
%\usetheme{Frankfurt}
%\usetheme{Goettingen}
%\usetheme{Hannover}
%\usetheme{Ilmenau}
%\usetheme{JuanLesPins}
%\usetheme{Luebeck}
%\usetheme{Madrid}
%\usetheme{Malmoe}
%\usetheme{Marburg}
%\usetheme{Montpellier}
%\usetheme{PaloAlto}
%\usetheme{Pittsburgh}
%\usetheme{Rochester}
%\usetheme{Singapore}
%\usetheme{Szeged}
%\usetheme{Warsaw}

% As well as themes, the Beamer class has a number of color themes
% for any slide theme. Uncomment each of these in turn to see how it
% changes the colors of your current slide theme.

%\usecolortheme{albatross}
%\usecolortheme{beaver}
%\usecolortheme{beetle}
%\usecolortheme{crane}
%\usecolortheme{dolphin}
%\usecolortheme{dove}
%\usecolortheme{fly}
%\usecolortheme{lily}
%\usecolortheme{orchid}
%\usecolortheme{rose}
%\usecolortheme{seagull}
%\usecolortheme{seahorse}
%\usecolortheme{whale}
%\usecolortheme{wolverine}

%\setbeamertemplate{footline} % To remove the footer line in all slides uncomment this line
%\setbeamertemplate{footline}[page number] % To replace the footer line in all slides with a simple slide count uncomment this line

%\setbeamertemplate{navigation symbols}{} % To remove the navigation symbols from the bottom of all slides uncomment this line
}

\usepackage{graphicx} % Allows including images
\usepackage{epstopdf}
\usepackage{booktabs} % Allows the use of \toprule, \midrule and \bottomrule in tables
\usepackage{multirow}
\usepackage[export]{adjustbox}
\usepackage{gensymb}
\renewcommand\footnoterule{{\color{black}\hrule height 0pt}}
\usepackage{appendixnumberbeamer} 
\usepackage{textgreek}
\usepackage{xcolor}
\usepackage{colortbl}
\usepackage{gensymb}
\graphicspath{{pics}}
\usepackage[normalem]{ulem}
\usepackage{mathtools}
\usepackage{amsmath}
\usepackage{amsfonts}

\newcommand{\overbar}[1]{\mkern 1.5mu\overline{\mkern-1.5mu#1\mkern-1.5mu}\mkern 1.5mu}

\setbeamertemplate{caption}{\raggedright\insertcaption\par}
\setbeamertemplate{navigation symbols}{}
\setbeamertemplate{itemize items}[circle]

\usepackage[absolute,overlay]{textpos}

%\AtBeginSection{\frame{\sectionpage}}

%----------------------------------------------------------------------------------------
%	TITLE PAGE
%----------------------------------------------------------------------------------------

\title[Conference or Meeting Title]{Presentation Title} % The short title appears at the bottom of every slide, the full title is only on the title page

\author{Your Name here}
\institute[SFU] % Your institution as it will appear on the bottom of every slide, may be shorthand to save space
{PhD Candidate \\ Supervisor: Dr. C. Andreoiu \\
Simon Fraser University \\ % Your institution for the title page
\medskip
\textit{email@sfu.ca} % Your email address
} % Date, can be changed to a custom date
\date{\today}



\begin{document}
%\begin{frame}
%\titlepage % Print the title page as the first slide
%\begin{figure}
%\includegraphics[width=0.5\textwidth]{logos.png}
%\end{figure}
%\end{frame}
 
 

%----------------------------------------------------------------------------------------
%	PRESENTATION SLIDES
%----------------------------------------------------------------------------------------

%------------------------------------------------



\begin{frame}
\frametitle{Varying the width of the Gaussian}
\begin{figure}
\includegraphics[width=\textwidth]{GaussianShapeParams-GaussianWidth.png}
\end{figure}
\end{frame}

\begin{frame}
\frametitle{Varying the amplitude - effect on HoO}
\begin{figure}
\includegraphics[width=\textwidth]{Gaussian-VaryShapeParams.png}
\end{figure}
\end{frame}

\begin{frame}
\frametitle{Varying the width - larger amplitude}
\begin{figure}
\includegraphics[width=\textwidth]{GaussianShapeParams-GaussianWidth-amp200.png}
\end{figure}
\end{frame}

\begin{frame}
\frametitle{Varying the position}
\begin{figure}
\includegraphics[width=0.9\textwidth]{GaussianShapeParams-GaussianWidth-PosVary.png}
\end{figure}
\end{frame}

\begin{frame}
\frametitle{Varying the amplitude}
\begin{figure}
\includegraphics[width=0.9\textwidth]{GaussianShapeParams-GaussianWidth-AmpVary.png}
\end{figure}
\end{frame}

\begin{frame}
\frametitle{HoO values - position and amplitude}
\begin{figure}
\includegraphics[width=0.9\textwidth]{GaussianShapeParams-GaussianWidth-AmpVary.png}
\end{figure}
\end{frame}

%
%%------------------------------------------------
%

%
%\appendix
%
%\begin{frame}
%\frametitle{Beginning of backup slides}
%\end{frame}


%-----------------------------------------------

\end{document} 
